\documentclass[12pt,a4paper,titlepage]{article}
\usepackage[utf8]{inputenc}
\usepackage[T1]{fontenc}
\usepackage[top=3cm, bottom=3cm, left=2cm, right=2cm]{geometry}
\usepackage{textcomp}
\usepackage{amsmath}
\usepackage{mathtools}
\usepackage{amsfonts}
\usepackage{amssymb}
\usepackage{amsthm}
\usepackage{titlesec}
\usepackage{fancyhdr}
\usepackage{lastpage}
\usepackage{fix-cm}
\usepackage{graphicx}
\usepackage{hyperref}
\usepackage{xcolor}
\usepackage{mdwlist}
\usepackage{listings}
\usepackage{float}
\usepackage{wrapfig}
\usepackage{datetime}
\usepackage[perpage,para,bottom,marginal]{footmisc}
\usepackage{listings}
\usepackage{caption}
\usepackage{enumitem}
\usepackage{multicol}
\usepackage[cmtip,all]{xy}
\newdateformat{dmny}{\monthname[\THEMONTH] \THEYEAR}
\newdateformat{dyo}{\THEYEAR}
\setlength{\headheight}{30pt}
\pagestyle{fancy}

\author{Nicolas Hafner}
\lhead{Nicolas Hafner}
\title{Analysis II}
\chead{Analysis II}
\rhead{Zürich, \dmny\today}
\cfoot{\thepage\ / \pageref{LastPage}}
\lfoot{\copyright \dyo\today TymoonNET/NexT}
\date{\d_mny\today}

\newcommand{\longsquiggly}{\xymatrix{{}\ar@{~>}[r]&{}}}
\renewcommand{\Re}{\operatorname{Re}}
\renewcommand{\Im}{\operatorname{Im}}
\renewcommand{\arg}{\operatorname{arg}}
\renewcommand{\d}{\partial}
\newcommand{\arsinh}{\operatorname{arsinh}}
\newcommand{\arcosh}{\operatorname{arcosh}}
\newcommand{\artanh}{\operatorname{artanh}}
\newcommand{\setC}{\mathbb{C}}
\newcommand{\setR}{\mathbb{R}}
\newcommand{\setZ}{\mathbb{Z}}
\newcommand{\setN}{\mathbb{Z}^{\geq0}}
\newcommand{\Graph}{\operatorname{Graph}}
\newcommand{\vol}{\operatorname{vol}}
\newcommand{\diam}{\operatorname{diam}}
\newcommand{\rot}{\operatorname{rot}}
\newcommand{\divv}{\operatorname{div}}

\newcount\colveccount
\newcommand*\colvec[1]{
  \global\colveccount#1
  \begin{pmatrix}
    \colvecnext
  }
  \def\colvecnext#1{
    #1
    \global\advance\colveccount-1
    \ifnum\colveccount>0
    \\
    \expandafter\colvecnext
    \else
  \end{pmatrix}
  \fi
}

\newcommand{\twovec}[2]{\mathop{\left(\substack{#1 \\ #2}\right)}}
\newcommand{\threevec}[3]{\mathop{\left(\substack{#1 \\ #2 \\ #3}\right)}}

\begin{document}	
\begin{center}{\bfseries\Huge Analysis II - 2014.05.26}\end{center}
\textit{Erinnerung}: Gauss \\
$\int_X\divv K \;d\vol_3=\int_{\d X}K\cdot n \;d\vol_2$ \\
\\
\textit{Beispiel}: $K(x)=\frac{cx}{|x|^3},\quad X=\left\{x\in\setR^3\middle||x|\leq R\right\},\quad \d X=\left\{x\in\setR^3\middle||x|= R\right\}$ \\
$n(x)=\frac{x}{|x|}$ für $x\in\d X$ \\
Rechte Seite: $=\int_{\d X}\frac{cx}{|x|^3}\cdot\frac{x}{|x|} \;d\vol_2 = \int_{\d X}\frac{cR^2}{R^4} \;d\vol_2 = \frac{c}{R^2}\underbrace{\vol_2(\d X)}_{4\pi R^2} = 4\pi c$ \\
$\divv K=\sum\frac{\d K_i}{\d x_i} = \frac{3x}{|x|^3}-\frac{3c|x|^2}{|x|^5}=0$ \\
$\frac{\d}{\d x_i}(\frac{cx_i}{|x|^3})=\frac{c}{|x|^3}-\frac{3cx_i}{|x|^5}$ \\
Also $K$ auf $\setR^3\setminus\{\threevec{0}{0}{0}\}$ divergenzfrei. Vorsicht: Gauss nicht anwendbar, weil $\threevec{0}{0}{0}\in X$ ist. Aber sei $Y\subset X$ mit $\threevec{0}{0}{0}\in Y^0 \Rightarrow$ Gauss anwendbar auf $X\setminus Y$. Dann folgt: \\
$\int_{X\setminus Y}\underbrace{\divv K}_{=0}\;d\vol_3=0 = \underbrace{\int_{\d X}K\cdot n\;d\vol-2}_{4\pi c}-\int_{\d Y}K\cdot n\;d\vol_2 \Rightarrow \int_{\d Y}K\cdot n\;d\vol_2=4\pi c$ \\
für jedes kompakte $Y\subset\setR^3$ mit $0\in Y^0$

\section*{Satz von Stokes}
Sei $B\subset\setR^2$ kompakt mit stückweise $C^1$-Rand, sei $\varphi:B\to F$ eine stückweise $C^1$-parametrisierte Fläche, sei $\d F:=\varphi(\d B)$. \\
\\
\textit{Definition}: Für $K$ Vektorfeld auf $U\subset\setR^3$ ist $\rot K:=\nabla\times K = \threevec{\d/\d x_1}{\d/\d x_2}{\d/\d x_3}\times\threevec{K_1}{K_2}{K_3}=\threevec{\frac{\d K_3}{\d x_2}-\frac{\d K_2}{\d x_3}}{\frac{\d K_1}{\d x_3}-\frac{\d K_3}{\d x_1}}{\frac{\d K_2}{\d x_1}-\frac{\d K_1}{\d x_2}}$ \\
\\
\textit{Bemerkung}: $f$ $C^2$-Skalarfeld $\Rightarrow\rot(\operatorname{grad} f)=0$ \\
\\
Sei $U\subset\setR^3$ offen mit $F\subset U$ \\
\\
\textit{Satz}: $$\int_F(\rot K)\cdot n \;d\vol_2 = \int_{\d F}K\cdot dx$$ \\
Green: $\int_B((\rot K)\circ\varphi)\cdot(\varphi_u\times\varphi_v)\;d\vol_2\twovec{u}{v}=\int_{\d B}(K\circ\varphi)\times\nabla\varphi\times d\twovec{u}{v}$ \\
\\
\textit{Bedeutung}: $\rot K$ misst die lokale Zirkulationsrate von $K$ in jeder der drei Raumrichtungen. \\
\\
\textit{Beispiel}: $K(x):=\omega\times x$ in $\setR^3$. Geschwindigkeitsfeld einer gleichmässigen Drehung um $\omega\setR$. \\
$\rot K=\threevec{\d_1}{\d_2}{\d_3}\times\threevec{\omega_1}{\omega_2}{\omega_3}\times\threevec{x_1}{x_2}{x_3} = [..] = \threevec{2\omega_1}{2\omega_2}{2\omega_3}=2\omega$ \\
\\
\\
\textit{Beispiel}: $F:=\left\{\threevec{x}{y}{z}\in\setR^3\middle|\substack{x^2+y^2+z^2=1 \\ z\geq 0}\right\}$ Orientierung von $F$: überall $n$ von 0 weg. \\
Sei $K\threevec{x}{y}{u}:=\threevec{y(z^2-x^2)}{x(y^2-z^2)}{z(x^2-y^2)} \Rightarrow \rot K=\threevec{\d_1}{\d_2}{d_3}\times K = \threevec{\d_y(z(x^2-y^2))-\d_z(x(y^2-z^2))}{..}{..}=[..]=\threevec{2z(x-y)}{2x(y-z)}{y^2+x^2-2z^2}$ \\
Apparently this result might be wrong. \\
\\
$\int_F\rot K\cdot n\;d\vol_2=\int_{\d F}K\cdot d\threevec{x}{y}{z}$ \\
Rechne rechte Seite. Für $\d F$ wähle Parametrisierung $\gamma:[0,2\pi]\to\d F, \quad t\mapsto\threevec{\cos t}{\sin t}{0}$ \\
$=\int_0^{2\pi}(K\circ\gamma)(t)\cdot\gamma'(t)\;dt=\int_0^{2\pi}\threevec{-\sin t\cos^2t}{\cos t\sin^2t}{0}\cdot\threevec{-\sin t}{\cos t}{0} \;dt = \int_0^{2\pi}2\sin^2t\cos^2t \;dt$ \\
$\int_0^{2\pi}2\frac{1-\cos^2t}{2}\frac{1+\cos^2t}{2} \;dt=\int_0^{2\pi}\frac{1}{2}(1-\cos^22t) \;dt = \int_0^{2\pi}\frac{1}{2}(1-\frac{1+\cos4t}{2}) \;dt = \frac{1}{2}(1-\frac{1}{2})2\pi=\frac{\pi}{2}$ \\
\\
\textit{Beispiel}: $K(x)=\frac{\omega\times x}{|\omega\times x|^2}$ auf $\setR^3_{\setminus\setR\omega} \quad \rot K=\threevec{0}{0}{0}$ hat lokal, aber nicht global, ein Potential. \\
\\
\textit{Fakt}: $\int_\gamma K\cdot dx=2\pi k$ für jede geschlossene Kurve $\gamma$ in $\setR^3_{\setminus\setR\omega}$ \\
\\
Seien zwei Kurven $\gamma_1,\gamma_2$ mit Windungszahl 1. Wähle $F$ so dass $F\subset\setR^3_{\setminus\setR\omega}$ \quad $\d F=\gamma_1-\gamma_2$. \\
Stokes $\Rightarrow \underbrace{\int_F\rot K\cdot n \;d\vol_2}_{=0} = \int_{\gamma_1}K\cdot dx-\int_{\gamma_2}K\cdot dx$ \\
Also $\int_{\gamma_1}K\cdot dx = \int_{\gamma_2}K\cdot dx$ \\
\\
\\
\textit{Spezialfall}: $\d F=\varnothing$. geschlossene Fläche. Wende Stokes auf Teilstücke an. \\
\\
\textit{Satz}: $\d F=\varnothing \Rightarrow \int_F\rot K\cdot n \;d\vol_2=0$

\end{document}