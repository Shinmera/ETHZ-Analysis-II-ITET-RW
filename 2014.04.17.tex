\documentclass[12pt,a4paper,titlepage]{article}
\usepackage[utf8]{inputenc}
\usepackage[T1]{fontenc}
\usepackage[top=3cm, bottom=3cm, left=2cm, right=2cm]{geometry}
\usepackage{textcomp}
\usepackage{amsmath}
\usepackage{mathtools}
\usepackage{amsfonts}
\usepackage{amssymb}
\usepackage{amsthm}
\usepackage{titlesec}
\usepackage{fancyhdr}
\usepackage{lastpage}
\usepackage{fix-cm}
\usepackage{graphicx}
\usepackage{hyperref}
\usepackage{xcolor}
\usepackage{mdwlist}
\usepackage{listings}
\usepackage{float}
\usepackage{wrapfig}
\usepackage{datetime}
\usepackage[perpage,para,bottom,marginal]{footmisc}
\usepackage{listings}
\usepackage{caption}
\usepackage{enumitem}
\usepackage{multicol}
\usepackage[cmtip,all]{xy}
\newdateformat{dmny}{\monthname[\THEMONTH] \THEYEAR}
\newdateformat{dyo}{\THEYEAR}
\setlength{\headheight}{30pt}
\pagestyle{fancy}

\author{Nicolas Hafner}
\lhead{Nicolas Hafner}
\title{Analysis II}
\chead{Analysis II}
\rhead{Zürich, \dmny\today}
\cfoot{\thepage\ / \pageref{LastPage}}
\lfoot{\copyright \dyo\today TymoonNET/NexT}
\date{\d_mny\today}

\newcommand{\longsquiggly}{\xymatrix{{}\ar@{~>}[r]&{}}}
\renewcommand{\Re}{\operatorname{Re}}
\renewcommand{\Im}{\operatorname{Im}}
\renewcommand{\arg}{\operatorname{arg}}
\renewcommand{\d}{\partial}
\newcommand{\arsinh}{\operatorname{arsinh}}
\newcommand{\arcosh}{\operatorname{arcosh}}
\newcommand{\artanh}{\operatorname{artanh}}
\newcommand{\setC}{\mathbb{C}}
\newcommand{\setR}{\mathbb{R}}
\newcommand{\setZ}{\mathbb{Z}}
\newcommand{\setN}{\mathbb{Z}^{\geq0}}
\newcommand{\Graph}{\operatorname{Graph}}
\newcommand{\vol}{\operatorname{vol}}
\newcommand{\diam}{\operatorname{diam}}

\newcount\colveccount
\newcommand*\colvec[1]{
  \global\colveccount#1
  \begin{pmatrix}
    \colvecnext
  }
  \def\colvecnext#1{
    #1
    \global\advance\colveccount-1
    \ifnum\colveccount>0
    \\
    \expandafter\colvecnext
    \else
  \end{pmatrix}
  \fi
}

\begin{document}	
\begin{center}{\bfseries\Huge Analysis II - 2014.04.17}\end{center}
\section*{Trägheitsmoment}
Punktmasse $m$ im Abstand $\rho$ zur Drehachse $\Rightarrow m\rho^2=m(x^2+y^2)$ \\
\textit{Speziell}: Drehung um $z$-Achse. \\
Körper mit Punktmenge $X\subset\setR^3$ kompakt und Massenverteilung \\
$\Theta_Z(X)=\int_X\mu\colvec{3}{x}{y}{z}(x^2+y^2) \;d\vol_3\colvec{3}{x}{y}{z}$ \\
\\
\\
\textit{Beispiel}: $X$ homogene Kugel mit Radius $R$ um $0$, $\mu$ konstant $\Rightarrow \Theta_Z(X)=|\text{Kugelkoordinaten}| $\\
$=\int\limits_{r=0}^R\int\limits_{\vartheta=-\frac{\pi}{2}}^{\frac{\pi}{2}}\int\limits_{\varphi=0}^{2\pi}\mu r^2\cos^2\vartheta r^2\cos\vartheta \;d\varphi d\vartheta dr = \mu\int\limits_0^Rr^4 \;dr \cdot\int\limits_{-\frac{\pi}{2}}^{\frac{\pi}{2}}\cos^3\vartheta \;d\vartheta \cdot\int\limits_0^{2\pi} \;d\varphi$ \\
Mittleres Integral: $\left|\substack{\sin\vartheta=s \\ \cos\vartheta d\vartheta=ds \\ \vartheta=\pm\frac{\pi}{2}\iff s=\pm 1 \\ \cos^2\vartheta=1-\sin^2\vartheta}\right| = \int\limits_{-1}^1(1-s^2) \;ds =(s-\frac{s^3}{3})\mid_{-1}^1=(1-\frac{1}{3})-(-1+\frac{1}{3})=\frac{4}{3}$ \\
$\Rightarrow =\mu\frac{R^5}{5}\cdot\frac{4}{3}\cdot 2\pi=\mu\frac{8\pi}{15}R^5$ \\
\\
\textit{Skalierung}: Sei $Y=\left\{\lambda\colvec{3}{x}{y}{z}\middle|\colvec{3}{x}{y}{z}\in X\right\}$ für $\lambda>0$. Beide $X,Y$ haben dieselbe konstante Massendichte $\mu$. $\Rightarrow\operatorname{Masse}(Y)=\int_Y\mu \;d\vol$. \\
Substituiere mit $\varphi:\setR^3\to\setR^3 \colvec{3}{x}{y}{z}\mapsto\lambda\colvec{3}{x}{y}{z}=\colvec{3}{\lambda x}{\lambda y}{\lambda z} \quad \det\nabla\varphi=\lambda^3$ \\
$\Rightarrow\operatorname{Masse}(Y)=\int_X\mu\lambda^3 \;d\vol \Rightarrow\operatorname{Masse}(Y)=\lambda^3\operatorname{Masse}(X)$ \\
$\Theta_Z(Y)=\int_Y\mu(x^2+y^2) \;d\vol\colvec{3}{x}{y}{z} = \int_X\mu(\lambda^2x^2+\lambda^2y^2)\lambda^3 \;d\vol\colvec{3}{x}{y}{z} \Rightarrow \Theta_Z(Y)=\lambda^5\Theta_Z(X)$
\newpage
\section*{Gravitation einer Kugelschale}
Sei $X:=\left\{x\in\setR^3\middle|R_1\leq x\leq R_2\right\}$ und $P=\colvec{3}{0}{0}{a}$. Gesucht: Anziehungskraft von $X$ auf $P$. \\
\textit{Newton}: Anziehungskraft einer Punktmasse $m'$ in $P'$ auf eine Punktmasse $m$ in $P$ gleich $Gmm'\frac{P'-P}{|P'-P|^3}$. $F=\int_X\mu(x)Gm\frac{x-P}{|x-P|^3} \;d\vol(x)$ \\
\textit{Von Oben}: $F=\int_X\underbrace{\mu Gm}_{\text{const}}\frac{\colvec{3}{x}{y}{z-a}}{\left|\colvec{3}{x}{y}{z-a}\right|^3} \;d\vol\colvec{3}{x}{y}{z} = \int\limits_{r=R_1}^{R_2}\int\limits_{\vartheta=-\frac{\pi}{2}}^{\frac{\pi}{2}}\int\limits_{\varphi=0}^{2\pi}\mu Gm \frac{\colvec{3}{r\cos\vartheta\cos\varphi}{r\cos\vartheta\sin\varphi}{r\sin\vartheta-a}}{(...)^{3/2}} r^2\cos\vartheta \;d\varphi d\vartheta dr$ \\
$... = r^2\cos^2\vartheta\cos^2\varphi+r^2\cos^2\vartheta\sin^2\varphi+(r\sin\vartheta-a)^2=r^2\cos^2\vartheta+r^2\sin^2\vartheta-2ra\sin\vartheta+a^2$ \\
$=r^2-2ra\sin\vartheta+a^2$ \\
\\
$\Rightarrow \left.\begin{array}{ll}
  \int_0^{2\pi}\cos\varphi \;d\varphi & =0 \\
  \int_0^{2\pi}\sin\varphi \;d\varphi & =0 \\
  \int_0^{2\pi}1 \;d\varphi & =2\pi
\end{array}\right\} = G\mu m2\pi\int\limits_{R_1}^{R_2}\int\limits_{-\frac{pi}{2}}^{\frac{\pi}{2}}\frac{\colvec{3}{0}{0}{r\sin\vartheta-a}}{(...)^{3/2}}r^2\cos\vartheta \;d\vartheta dr = \int\limits_{R_1}^{R_2}\int\limits_{-\frac{pi}{2}}^{\frac{\pi}{2}}\frac{(r\sin\vartheta-a)r^2\cos\vartheta}{(r^2-2ra\sin\vartheta+a^2)^{3/2}} \;d\vartheta dr$ \\
$=\left|\substack{\sin\vartheta=s \\ \cos\vartheta \;dvartheta = ds}\right| = \int\limits_{R_1}^{R_2}\left(\int\limits_{-1}^1\frac{(rs-a)r^2}{(r^2-2ras+a^2)^{3/2}}\right) \;dr = \left|\substack{u^2=r^2-2ras+a^2,\;u\geq 0 \\ 2u \;du = -2ra \;ds \\ s=-1 \iff u^2=r^2+2ra+a^2=(r+a)^2 \iff u=|r+a| \\ s=1 \iff u^2=r^2-2ra+a^2=(r-a)^2 \iff u=|r-a| \\ 2ras = r^2+a^2-u^2 \\ rs-a=\frac{r^2+a^2-u^2}{2a}-a = \frac{r^2-a^2-u^2}{2a}}\right| $ \\
$=\int\limits_{R_1}^{R_2}\left(\int\limits_{|r+a|}^{|r-a|}\frac{(\frac{r^2-a^2-u^2}{2a})r^2}{u^3}\frac{2u \;du}{-2ra}\right) \;dr = \int\limits_{R_1}^{R_2}\left(\int\limits_{|r+a|}^{|r-a|}\frac{r^2}{-2ra^2}(\frac{r^2-a^2-u^2}{u^2}) \;du\right) \;dr = \int\limits_{R_1}^{R_2}\left(\left.\frac{r^2}{-2ra^2}(\frac{a^2-r^2}{u}-u)\right|_{u=|r+a|}^{u=|r-a|}\right) \;dr$ \\
$\int\limits_{R_1}^{R_2}\frac{-r}{2a^2}\left(\substack{\frac{a^2-r^2}{|r-a|}-|r-a| \\ -\frac{a^2-r^2}{|r+a|}+|r+a|}\right) \;dr \Rightarrow \left(\substack{\frac{(a-r)(a+r)}{|a-r|}-|a-r| \\ -(a-r)+(a+r)}\right) = \left\{\begin{array}{lll}
  a>r &\Rightarrow \left(\substack{(a+r)-(a-r) \\ -(a-r)+(a+r)}\right) &=4r \\
  a<r &\Rightarrow \left(\substack{-(a+r)+(a-r) \\ -(a-r)+(a+r)}\right) &=0
\end{array}\right.$ \\
$\begin{array}{lll}
 \text{Für}\; a<R_1 &\;\text{folgt:}\; ..=0 &\Rightarrow =\colvec{3}{0}{0}{0} \\
 \text{Für}\; a>R_2 &\;\text{folgt:}\; ..=\int\limits_{R_1}^{R_2}\frac{-r}{2a^2}4r \;dr=\left.\frac{-2}{a^2}\frac{r^3}{3}\right|_{R_1}^{R_2}=-\frac{2}{3a^2}(R_2^3-R_1^3) &\Rightarrow =\colvec{3}{0}{0}{G\mu m2\pi\frac{-2}{3a^2}(R_2^3-R_1^3)}
\end{array}$ \\
Masse von $X=:M$. \quad $\mu\vol(X) = \mu\frac{4\pi}{3}(R_2^3-R_1^3) \Rightarrow \text{Kraft}=\colvec{3}{0}{0}{-GmM\frac{1}{a^2}}$ \\
= die Kraft, die eine Punktmasse $M$ im Schwerpunkt von $X$ auf $P$ auswirkt.

\end{document}