\documentclass[12pt,a4paper,titlepage]{article}
\usepackage[utf8]{inputenc}
\usepackage[T1]{fontenc}
\usepackage[top=3cm, bottom=3cm, left=2cm, right=2cm]{geometry}
\usepackage{textcomp}
\usepackage{amsmath}
\usepackage{mathtools}
\usepackage{amsfonts}
\usepackage{amssymb}
\usepackage{amsthm}
\usepackage{titlesec}
\usepackage{fancyhdr}
\usepackage{lastpage}
\usepackage{fix-cm}
\usepackage{graphicx}
\usepackage{hyperref}
\usepackage{xcolor}
\usepackage{mdwlist}
\usepackage{listings}
\usepackage{float}
\usepackage{wrapfig}
\usepackage{datetime}
\usepackage[perpage,para,bottom,marginal]{footmisc}
\usepackage{listings}
\usepackage{caption}
\usepackage{enumitem}
\usepackage{multicol}
\usepackage[cmtip,all]{xy}
\newdateformat{dmny}{\monthname[\THEMONTH] \THEYEAR}
\newdateformat{dyo}{\THEYEAR}
\setlength{\headheight}{30pt}
\pagestyle{fancy}

\author{Nicolas Hafner}
\lhead{Nicolas Hafner}
\title{Analysis II}
\chead{Analysis II}
\rhead{Zürich, \dmny\today}
\cfoot{\thepage\ / \pageref{LastPage}}
\lfoot{\copyright \dyo\today TymoonNET/NexT}
\date{\d_mny\today}

\newcommand{\longsquiggly}{\xymatrix{{}\ar@{~>}[r]&{}}}
\renewcommand{\Re}{\operatorname{Re}}
\renewcommand{\Im}{\operatorname{Im}}
\renewcommand{\arg}{\operatorname{arg}}
\renewcommand{\d}{\partial}
\newcommand{\arsinh}{\operatorname{arsinh}}
\newcommand{\arcosh}{\operatorname{arcosh}}
\newcommand{\artanh}{\operatorname{artanh}}
\newcommand{\setC}{\mathbb{C}}
\newcommand{\setR}{\mathbb{R}}
\newcommand{\setZ}{\mathbb{Z}}
\newcommand{\setN}{\mathbb{Z}^{\geq0}}
\newcommand{\Graph}{\operatorname{Graph}}

\newcount\colveccount
\newcommand*\colvec[1]{
  \global\colveccount#1
  \begin{pmatrix}
    \colvecnext
  }
  \def\colvecnext#1{
    #1
    \global\advance\colveccount-1
    \ifnum\colveccount>0
    \\
    \expandafter\colvecnext
    \else
  \end{pmatrix}
  \fi
}

\begin{document}	
\begin{center}{\bfseries\Huge Analysis II - 2014.04.03}\end{center}
\textit{Erinnerung}: $U\subset\setR^n\overset{f}{\to}\setR^m$ diff'bar in $\xi\in U \iff f(x)=f(\xi)+\underbrace{\nabla f(\xi)}_{\mathclap{\substack{\text{Funktionalmatrix} \\ \text{Jacobi-Matrix}}}}\overbrace{(x-\xi)}^{\mathclap{\text{Spaltenvektor}}}+o(|x-\xi|)$ für $x\to\xi$. \\
\\
\textit{Vergleich}: $m=1$, $f$ zwei mal stetig diff'bar \\
$\longsquiggly f(x)=f(\xi)+\nabla f(\xi)(x-\xi)+\frac{1}{2}(x-\xi)^T\underbrace{\nabla^2f(\xi)}_{\mathclap{\text{Hesse-Matrix}}}(x-\xi)+o(|x-\xi|^2)$ \\
\\
\textit{Definition}: $f$ ist regulär in $\xi$ wenn $\operatorname{Rang}(\nabla f(\xi))=\min(n,m)$ ist. \\
\\
\textit{Beispiel}: $f:\setR^2\to\setR^3,\colvec{2}{r}{t}\mapsto\colvec{3}{r\cos t}{r\sin t}{t}$ \\
$\nabla f=\begin{pmatrix}
  \cos t & -r\sin t \\
  \sin t & r\cos t \\
  0 & 1
\end{pmatrix} \Rightarrow \left.\begin{array}{l}
  f \;\text{überall regulär} \\
  f \;\text{injektiv} \\
\end{array}\right\}\Rightarrow \operatorname{Bild}(f)$ überall reguläre Fläche. \\
\\
\\
\textit{Satz}: Seien $X\subset\setR^n\overset{f}{\to}Y\subset\setR^m\overset{g}{\to}Z\subset\setR^l$ dann ist $g\circ f$ diff'bar und \\
$\nabla(g\circ f)(\xi)=\underbrace{\nabla g}_{\mathclap{n\times m\text{-matrix}}}(f(\xi))\underbrace{\nabla f}_{\mathclap{m\times l\text{-matrix}}}(\xi)$. Dies ist die \emph{Kettenregel}. \\
\\
\textit{Beweis}:
\begin{align*}
  g(f(x)) &=g(f(\xi)+\nabla f(\xi)(x-\xi)+o(|x-\xi|)) = \\
  &= g(f(\xi))+\nabla g(f(\xi) (\nabla f(\xi)(x-\xi)+o(|x-\xi|)) + o(|\nabla f(\xi)(x-\xi)+o(|x-\xi|)) \\
  &= g(f(\xi))+(\nabla g(f(\xi))\nabla f(\xi))(x-\xi)+o(|x-\xi|)
\end{align*} \\
\\
\textit{Beispiel}: $\begin{array}{l}
  f:\colvec{2}{r}{\varphi}\mapsto\colvec{2}{r\cos\varphi}{r\sin\varphi} \\
  g:\colvec{2}{x}{y}\mapsto\colvec{2}{\sqrt{x^2+y^2}}{\arg(x+iy)}
\end{array} \Rightarrow
\begin{array}{ll}
  \nabla f=\begin{pmatrix}
    \cos\varphi & -r\sin\varphi \\ 
    \sin\varphi & r\cos\varphi 
  \end{pmatrix} \\
  \nabla g=\begin{pmatrix}
    \frac{x}{r} & \frac{y}{r} \\ 
    \frac{-y}{r} & \frac{x}{r} 
  \end{pmatrix} & = \begin{pmatrix}
    \cos\varphi & \sin\varphi \\
    \frac{-\sin\varphi}{r} & \frac{\cos\varphi}{r}
  \end{pmatrix}
\end{array}\\
\left.\begin{array}{l}
  f\circ g=id \\
  g\circ f=id
\end{array}\right\}\Rightarrow\left\{\begin{array}{l}
  \nabla f\nabla g=\nabla id \\
  \nabla g\nabla f=\nabla id
\end{array}\right\}=\begin{pmatrix} 1&0\\0&1 \end{pmatrix}$ \\
\\
\newpage
\textit{Nachtrag} zur Kettenregel: $f,g$ $k$-mal stetig diff'bar $\Rightarrow g\circ f$ dito. \\
\\
\textit{Satz}: Seien $X,Y\subset\setR^n$ offen, $f:X\to Y$ $k$-mal stetig diff'bar, $\xi\in X$ und $\eta:=f(\xi)$.
\begin{enumerate}[label=(\alph*)]
  \item Ist $f$ invertierbar unf $f^{-1}$ diff'bar, so ist $\nabla f(x)$ invertierbar und $\nabla (f^{-1})(f(x)))=(\nabla f(x))^{-1}$. Ausserdem ist dann $g$ ebenfalls $k$-mal stetig diff'bar.

  \item Ist $\nabla f(\xi)$ invertierbar, so existieren offene $U\subset X$ und $V\subset Y$ mit $\xi\in U$ und $\eta\in V$, so dass $f$ eine bijektive Abbildung $f\mid U:U\to V$ induziert deren Inverse (a) erfüllt.
  \end{enumerate}
.\\
\textit{Beispiel}: $f:\setR\to\setR\; x\mapsto x^3$ bijektiv, aber $f^{-1}(y)=\sqrt[3]{y}$ in $y=0$ nicht diff'bar. \\
\\
\textit{Zu (a)}: Kettenregel: $f^{-1}\circ f=id \Rightarrow (\nabla f^{-1})(f(x))\nabla f(x)=\nabla (f^{-1}\circ f)(x)=\nabla (id)(x)=I_n$ \\
\\
\textit{Zu (b)}: $\nabla f(\xi)$ invertierbar $\iff \overbrace{\det\nabla f(\xi)}^{\text{stetig}}\neq 0 $\\
$\iff \det\nabla f(x) \neq 0$ in der Nähe von $\xi \iff \nabla f(x)$ invertierbar nahe $\xi$. \\
\\
\textit{Beispiel}: $f:\setC\to\setC,\; z\mapsto z^2\quad \setR^2\to\setR^2,\colvec{2}{x}{y}\mapsto\colvec{2}{x^2-y^2}{2xy} \Rightarrow \nabla f=\begin{pmatrix} 2x & -2y \\ 2y & 2x \end{pmatrix} $ \\
$\det\nabla f=4(x^2+y^2)=0 \iff x=y=0$ \\
\\
\textit{Definition}: $\det\nabla f$ heisst \emph{Funktionaldeterminante}. \\
\\
\textit{Geometrische Bedeutung}: $f(x)=\overbrace{f(\xi)}^{\mathclap{\text{Translation}}}+\overbrace{\nabla f(\xi)(x-\xi)}^{\mathclap{\text{Koordinatenwchsel}}}+o(|x-\xi|)$ \\
$\nabla f(\xi)$ bildet einen Quader mit Kanten $h_ie_i$ auf einem Raumspat mit Kanten $\nabla f(\xi)\cdot h_ie_i$ ab. Das Volumen des Bilds ist $\det(\nabla f(\xi))\cdot h_1..h_n$. $\quad\operatorname{vol}(f(Q))=|\det(\nabla f(\xi))|\cdot\operatorname{vol}(Q)$ \\
\\
\textit{Allgemein}: $|\det(\nabla f(\xi))|$ ist der lokale Volumenfaktor von $f$ bei $\xi$, das heisst für jedes $\epsilon>0$ und jede hinreichend gute Teilmenge $A\subset X\cap B_\epsilon(\xi)$ gilt: $\operatorname{vol}(f(A))=|\det(\nabla f(\xi))|\cdot\operatorname{vol}(A)+o(\epsilon)$

\end{document}