\documentclass[12pt,a4paper,titlepage]{article}
\usepackage[utf8]{inputenc}
\usepackage[T1]{fontenc}
\usepackage[top=3cm, bottom=3cm, left=2cm, right=2cm]{geometry}
\usepackage{textcomp}
\usepackage{amsmath}
\usepackage{mathtools}
\usepackage{amsfonts}
\usepackage{amssymb}
\usepackage{amsthm}
\usepackage{titlesec}
\usepackage{fancyhdr}
\usepackage{lastpage}
\usepackage{fix-cm}
\usepackage{graphicx}
\usepackage{hyperref}
\usepackage{xcolor}
\usepackage{mdwlist}
\usepackage{listings}
\usepackage{float}
\usepackage{wrapfig}
\usepackage{datetime}
\usepackage[perpage,para,bottom,marginal]{footmisc}
\usepackage{listings}
\usepackage{caption}
\usepackage{enumitem}
\usepackage{multicol}
\usepackage[cmtip,all]{xy}
\newdateformat{dmny}{\monthname[\THEMONTH] \THEYEAR}
\newdateformat{dyo}{\THEYEAR}
\setlength{\headheight}{30pt}
\pagestyle{fancy}

\author{Nicolas Hafner}
\lhead{Nicolas Hafner}
\title{Analysis II}
\chead{Analysis II}
\rhead{Zürich, \dmny\today}
\cfoot{\thepage\ / \pageref{LastPage}}
\lfoot{\copyright \dyo\today TymoonNET/NexT}
\date{\d_mny\today}

\newcommand{\longsquiggly}{\xymatrix{{}\ar@{~>}[r]&{}}}
\renewcommand{\Re}{\operatorname{Re}}
\renewcommand{\Im}{\operatorname{Im}}
\renewcommand{\arg}{\operatorname{arg}}
\renewcommand{\d}{\partial}
\newcommand{\arsinh}{\operatorname{arsinh}}
\newcommand{\arcosh}{\operatorname{arcosh}}
\newcommand{\artanh}{\operatorname{artanh}}
\newcommand{\setC}{\mathbb{C}}
\newcommand{\setR}{\mathbb{R}}
\newcommand{\setZ}{\mathbb{Z}}
\newcommand{\setN}{\mathbb{Z}^{\geq0}}
\newcommand{\Graph}{\operatorname{Graph}}

\begin{document}	
\begin{center}{\bfseries\Huge Analysis II - 2014.03.31}\end{center}
\textit{Beispiel}: Bestimme das Maximum von $f(x_1..x_n):=x_1\cdot..\cdot x_n$ auf \\
$B:=\{(x_1..x_n)\in\setR^n\mid \;\text{alle}\; x_i\geq 0\quad x_1+..+x_n=s\}$ für $s>0$ fest. \\
\\
\textit{Lösung}: Nebenbedingung $g(x_1..x_n):=x_1+..+x_n-s$. $B$ kompakt, da abgeschlossen, beschränkt und nichtleer. $f$ stetig $\Rightarrow$ Maximum existiert. Weil z.B. $(\frac{s}{n}..\frac{s}{b})\in B$ mit $f('')>0$ \\
$\Rightarrow$ Maximalstelle erfüllt $x_i>0$ für alle $i$. \\
$\Rightarrow$ lokales Extremum von $f$ unter $g=0$. \\
$\Rightarrow$ entspricht kritischem Punkt von $F(x_1..x_n):=x_1\cdot..\cdot x_n-\lambda(x_1+..+x_n-s)$ \\
$\Rightarrow \frac{\d F}{\d x_i}=x_1\cdot..\cdot \widehat{x_i}\cdot..\cdot x_n-\lambda\cdot 1=0$ \\
$\Rightarrow \frac{\d F}{\d \lambda}=-(x_1+..+x_n-s)=0 \iff \frac{x_1..x_n}{x_i}=\lambda \iff x_i=\frac{x_1..x_n}{\lambda} \iff $ alle $x_i$ gleich $\frac{s}{n}$. \\
$\Rightarrow$ Die einzige Maximalstelle von $f$ auf $B$ ist $(\frac{s}{n}..\frac{s}{n})$ mit $f(..)=\left(\frac{s}{n}\right)^n$ \\
$\Rightarrow \forall s\geq 0\forall x_1..x_n\geq 0:x_1+..+x_n=s:x_1\cdot..\cdot x_n\leq(\frac{s}{n})^n$ \\
\\
\textit{Folge}: Für alle $x_1..x_n\geq 0$ gilt $\underbrace{\sqrt[n]{(x_1\cdot..\cdot x_n)}}_{\text{geometrisches Mittel}}\leq\underbrace{\frac{x_1+..+x_n}{n}}_{\text{arithmetisches Mittel}}$. \\
\\
\textit{Beispiel}: Bestimme die Extrema von $f(x,y,z):=x$ auf \\
$B:=\{(x,y,z)\in\setR^3\mid x^2+y^2+z^2=1\quad 5x+4y+3z=0\}$ \\
\\
\textit{Lösung}: $g_1(x,y,z):=x^2+y^2+z^2-1 \quad g_2(x,y,z):=5x+4y+3z$ \\
$\Rightarrow \nabla g_1=(2x,2y,2z) \quad \nabla g_2=(5,4,3)$ \\
Wenn die linear abhängig sind, ist $(2x,2y,2z)=\lambda(5,4,3)$ sein. \\
$\Rightarrow g_2(x,y,z)=\frac{\lambda}{2}(5^2+4^2+3^2)=0 \Rightarrow \lambda=0 \Rightarrow x=y=z=0 \Rightarrow g_2(x,y,z)\neq 0$ \\
$\Rightarrow$ $B$ überall reguläre Kurve. \\
$F(x,y,z):=x+\lambda(x^2+y^2+z^2-1)+\mu(5x+4y+3z)$ \\
$\begin{array}{lll}
  F_x= & 1+2\lambda x+5\mu & \overset{!}{=}0 \\
  F_y= & 2\lambda y+4\mu & \overset{!}{=}0 \\
  F_z= & 2\lambda z+3\mu & \overset{!}{=}0 \\
  F_\lambda= & x^2+y^2+z^2-1 & \overset{!}{=}0 \\
  F_\mu= & 5x+4y+3z & \overset{!}{=}0 \\
\end{array} \Rightarrow \begin{array}{l}
  x=\frac{-1-5\mu}{2\lambda} \\
  y=\frac{-4\mu}{2\lambda} \\
  z=\frac{-3\mu}{2\lambda} \\
  \lambda\neq 0 \\
  \quad
\end{array}$ Einsetzen $\longsquiggly \lambda=\pm\frac{\sqrt{2}}{10} $ \\
$\Rightarrow (x,y,z)=\pm\frac{\sqrt{2}}{10}(-5,4,3)$. Dort ist $f=\pm\frac{\sqrt{2}}{2} \Rightarrow$ Diese Punkte sind die Extremalstellen.

\section*{Vektorwertige Funktionen}
Ab jetzt \emph{Spaltenvektoren}. Sei $U\subset\setR^n$ offen und $f:=(f_1..f_m)^T:U\to\setR^n$ \\
\\
\textit{Definition}: $f$ heisst $k$-mal diff'bar, bzw. $k$-mal stetig diff'bar, wenn jedes $f_i$ es ist. \\
\\
$f_i$ diff'bar in $\xi \iff f_i(x)=f_i(\xi)+\overbrace{\nabla f_i(\xi)}^{\text{Zeile}}\cdot \overbrace{(x-\xi)}^{\text{Spalte}}+o(|x-\xi|)$ \\
$f$ diff'bar in $\xi \iff f(x)=f(\xi)+\nabla f(\xi)\cdot (x-\xi)+o(|x-\xi|)$ \\
Dabei ist $\nabla f=\begin{pmatrix}\nabla f_1 \\ \vdots \\ \nabla f_m\end{pmatrix}=
\begin{pmatrix}
  \frac{\d f_1}{\d x_1} & \hdots & \frac{\d f_1}{\d x_n} \\
  \vdots & \ddots & \vdots \\
  \frac{\d f_m}{\d x_1} & \hdots & \frac{\d f_m}{\d x_n}
\end{pmatrix}$ \emph{erste Ableitung von $f$}, \emph{Funktionalmatrix von $f$}, \emph{Jacobi-Matrix}. \\
\\
\textit{Beispiel}: $f(x):=\overbrace{A}^{\mathclap{m\times n\;\text{Matrix}}}x+\underbrace{b}_{\mathclap{\text{Spaltenvektor der Länge}\;m}}$ \\
$f:\setR^n\to\setR^m$ Für jedes $\xi\in\setR^n$ ist $f(x)=(\underbrace{A\xi+b}_{f(\xi)})+\underbrace{A(x-\xi)}_{\nabla f(\xi)} \Rightarrow f$ diff'bar mit $\nabla f=A$. \\
\\
\textit{Beispiel}: $f:\setR^n\to\setR^n, x\mapsto x \Rightarrow \nabla f=\begin{pmatrix}1 & \hdots & 0 \\ \vdots & \ddots & \vdots \\ 0 & \hdots & 1\end{pmatrix}=: I_n$ \\
\\
\textit{Definition}: Eine diff'bare Funktion $f$ heisst \emph{regulär in $\xi$}, wenn $\nabla f(\xi)$ maximalen Rang hat. \\
\\
\textit{Spezialfall}: $m\leq n \Rightarrow $ Zeilen von $\nabla f$ lin. unabhängig. \\
Bedeutung: Niveaumenge $\{x\in U\mid f(x)=c\in\setR^m\}$ regulär. \\
\\
\textit{Spezialfall}: $m\geq n \Rightarrow $ Spalten von $\nabla f$ lin. unabhängig. \\
Bedeutung: $f$ lokal injektiv. $\operatorname{Bild}(f)$  lokal glatte Teilmenge der Dimension $n$. \\
\\
\textit{Beispiel}: $\setR\to\setR^2,\; t\mapsto\begin{pmatrix}t^2\\t^3\end{pmatrix},\; \nabla t=\begin{pmatrix}2t \\ 3t^2\end{pmatrix}\;$ regulär $\iff t\neq 0$ \\
\\
\textit{Vergleiche}: $\setR\to\setR^3,\; t\mapsto\begin{pmatrix}t\\t^2\\t^3\end{pmatrix}$
\end{document}