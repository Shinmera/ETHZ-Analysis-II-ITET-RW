\documentclass[12pt,a4paper,titlepage]{article}
\usepackage[utf8]{inputenc}
\usepackage[T1]{fontenc}
\usepackage[top=3cm, bottom=3cm, left=2cm, right=2cm]{geometry}
\usepackage{textcomp}
\usepackage{amsmath}
\usepackage{amsfonts}
\usepackage{amssymb}
\usepackage{amsthm}
\usepackage{titlesec}
\usepackage{fancyhdr}
\usepackage{lastpage}
\usepackage{fix-cm}
\usepackage{graphicx}
\usepackage{hyperref}
\usepackage{xcolor}
\usepackage{mdwlist}
\usepackage{listings}
\usepackage{float}
\usepackage{wrapfig}
\usepackage{datetime}
\usepackage[perpage,para,bottom,marginal]{footmisc}
\usepackage{listings}
\usepackage{caption}
\usepackage{enumitem}
\usepackage{multicol}
\usepackage[cmtip,all]{xy}
\newdateformat{dmny}{\monthname[\THEMONTH] \THEYEAR}
\newdateformat{dyo}{\THEYEAR}
\setlength{\headheight}{30pt}
\pagestyle{fancy}

\author{Nicolas Hafner}
\lhead{Nicolas Hafner}
\title{Analysis II}
\chead{Analysis II}
\rhead{Zürich, \dmny\today}
\cfoot{\thepage\ / \pageref{LastPage}}
\lfoot{\copyright \dyo\today TymoonNET/NexT}
\date{\d_mny\today}

\newcommand{\longsquiggly}{\xymatrix{{}\ar@{~>}[r]&{}}}
\renewcommand{\Re}{\operatorname{Re}}
\renewcommand{\Im}{\operatorname{Im}}
\renewcommand{\arg}{\operatorname{arg}}
\newcommand{\arsinh}{\operatorname{arsinh}}
\newcommand{\arcosh}{\operatorname{arcosh}}
\newcommand{\artanh}{\operatorname{artanh}}
\newcommand{\setC}{\mathbb{C}}
\newcommand{\setR}{\mathbb{R}}
\newcommand{\setZ}{\mathbb{Z}}
\newcommand{\setN}{\mathbb{Z}^{\geq0}}

\begin{document}	
\begin{center}{\bfseries\Huge Analysis II - 2014.03.03}\end{center}
\textit{Erinnerung}: Sei $L=(\frac{d}{dx})^n+a_1...+a_{n-1}\frac{d}{dx}+a_n$ für Koeffizienten $a_1..a_n$ \\
$f_L(T)=T^n+a_1T^{n-1}+...a_{n-1}T+a_n$.
$L(x^ne^{\lambda x})=\left\{\begin{array}{l l}
    0 & \quad \text{falls}\; \lambda\; \text{EW der Mult.}\; >n. \\
    ((\neq 0)x^{n-\mu})e^{\lambda x} & 
  \end{array}\right.$ \\
\textit{Z.B.} $Ly=e^{\lambda x}$ hat Lösung $cx^\mu e^{\lambda x}$.\\
\\
\textit{Fakt}: Ist $\lambda\in\setC$ Nullstelle von $f_L$ der Multiplizität $\mu\geq 0$ und $g(x)$ ein Polynom vom Grad $m$, so hat $Ly=g(x)e^{\lambda x}$ eine Lösung $f(x)e^{\lambda x}$ für ein Polynom vom Grad $m+\mu$. \\
\\
\textit{Beispiel}: $y^{(5)}+y=xe^{-x} \Rightarrow f_L(T)=T^5+1=0 \Rightarrow$ Nullstellen $-e^{\frac{2\pi ik}{5}},\;k\in\setZ \Rightarrow y=(ax^2+bx)e^{-x}$ für Konstanten $a,b$. \\
$\Rightarrow y'=(-ax^2+(2a-b)x+b)e^{-x}$ \\
$\Rightarrow y''=(ax^2+(b-4a)x+(2a-2b))e^{-x}$ \\
$\Rightarrow y'''=(-ax^2+(6a-b)x+(3b-6a))e^{-x}$ \\
$\Rightarrow y''''=(ax^2+(b-8a)x+(12a-4b))e^{-x}$ \\
$\Rightarrow y'''''=(-ax^2+(10a-b)x+(5b-20a))e^{-x}$ \\
$\implies y^{(5)}+y=((10a)x+(5b-20a))e^{-x} \overset{!}{=}xe^{-x} \iff \left\{\begin{array}{l}10a=1 \\ 5b-20a=0\end{array}\right\} \Rightarrow y=\frac{x^2}{10}+\frac{2x}{5}e^{-x}$\\
\\
\textit{Beispiel}: $\ddot y+\omega^2y=\cos\lambda t = \frac{e^{i\lambda t}+e^{-i\lambda t}}{2}$ angeregter harmonischer Oszillator, $\omega\neq 0$. Homogene Gleichung Fundamentallösungen $\cos\omega t,\; \sin\omega t$. Eigenwerte: $\pm \omega i$. \\
$\lambda\neq\pm\omega \Rightarrow y=ae^{i\lambda t}+be^{-i\lambda t}=(a+b)\cos\lambda t+(a-b)i\sin\lambda t$ für $a,b\in\setC$ \\
$y=c\cos\lambda t+d\sin\lambda t$ \\
$\Rightarrow \ddot y=-c\lambda^2\cos\lambda t-d\lambda^2\sin\lambda t $ \\
$\implies \ddot y+\omega^2y=(\omega^2-\lambda^2)(\cos\lambda t+d\sin\lambda t) \overset{!}{=} \cos\lambda t \iff \left\{\begin{array}{l}(\omega^2-\lambda^2)c=1 \\ (\omega^2-\lambda^2)d=0\end{array}\right\} \Rightarrow y=\frac{\cos\lambda t}{\omega^2-\lambda^2}$ \\
Fall $w=\pm\lambda$: Ansatz: $y=ct\cos\lambda t+dt\sin\lambda t $ \\
$\Rightarrow \dot y=c\cos\lambda t-c\lambda t\sin\lambda t+d\sin\lambda t+d\lambda t\cos\lambda t$\\
$\Rightarrow \ddot y=-2c\lambda\sin\lambda t-c\lambda^2t\cos\lambda t+2d\lambda\cos\lambda t-d\lambda^2t\sin\lambda t$ \\
$\implies \ddot y+\omega^2y=\ddot y+\lambda^2 y=-2c\lambda\sin\lambda t+2d\lambda\cos\lambda t \overset{!}{=} \cos\lambda t \iff \left\{\begin{array}{l}c=0 \\ d=\frac{1}{2}\lambda\end{array}\right\} \Rightarrow y=\frac{1}{2\lambda}t\sin\lambda t$

\section*{Systeme von Differentialgleichungen / Gekoppelte DGL}
\textit{Fakt}: Existenz und Eindeutigkeitssatz gilt genauso. \\
\\
\textit{Beispiel}: $\setR^2\simeq\setC$ \\
\\
\textit{Spezialfall}: Konstante Koeffizienten, $n=1$, linear, homogen. \\
$\Rightarrow y_1'=a_{11}y_1+..+a_{1m}y_m \quad..\quad y_m'=a_{m1}y_1+..+a_{mm}y_m$ \\
Mit $y=(y_1..y_m)^T$ ist dies äquivalent zu $y'=Ay$ mit $A=\begin{pmatrix}a_{11} & \cdots & a_{1m} \\
  \cdots & \ddots  & \cdots \\
  a_{m1} & \cdots & a_{mm} \end{pmatrix}$ \\
Für jedes $v\in\setC^m\neq 0$ und $\lambda\in\setC$ gilt $y:=ve^{\lambda x} \Rightarrow y'=\lambda ve^{\lambda x}\overset{?}{=} Ay=Ave^{\lambda x} \iff \lambda v=Av$ \\
Also ist $ve^{\lambda x}$ eine Lösung $\iff \lambda$ Eigenwert von $A$ und $v$ ein zugehöriger Eigenvektor. \\
\\
\textit{Folge}: Ist $A$ diagonalisierbar und $v_1..v_m$ eine Basis von $\setC^m$ bestehend aus Eigenvektoren zu den Eigenwerten $\lambda_1..\lambda_m\in\setC$, so ist $v_1e^{\lambda_1x}..v_me^{\lambda_mx}$ eine Basis des Lösungsraums. \\
\\
\textit{Beispiel}: $y_1'=-y_1+3y_2,\; y_2'=2y_1-2y_2$ mit $y_1(0)=5,\; y_2(0)=0$ \\
\textit{Lösung}: $A=\begin{pmatrix}-1&3 \\ 2&-2\end{pmatrix} \Rightarrow det(A)=(-1-T)(-2-T)-2\cdot 3 = T^2+3T-4=(T-1)(T+4)$ \\
$\Rightarrow $ Eigenwerte: $\lambda=\{1,-4\}$ Eigenvektoren: $v=\{\begin{pmatrix}3&2\end{pmatrix}^T, \begin{pmatrix}-1&1\end{pmatrix}^T\}$ \\
$\Rightarrow y=av_1e^x+bv_2e^{-4x} \Rightarrow y_1=3ae^x-be^{-4x},\quad y_2=2ae^x+be^{-4x}$ \\
$\Rightarrow \left.\begin{array}{l}
  y_1(0)=3a-b=5 \\
  y_2(0)=2a+b=0
\end{array}\right| \iff \left|\begin{array}{l}
  a=1 \\ b=-2
\end{array}\right\} \Rightarrow y_1=3e^x+2e^{-4x},\quad y_2=2e^x-2e^{-4x}$ \\
\\
\textit{Anmerkung}: $y^{(n)}+a_1y^{(n-1)}+..+a_0y=0 \Rightarrow$ Setze $y_i:=y^{(i-1)}$ für $i=1..n $ \\
$\Rightarrow y_i'=y_{i+1}$ für $1\leq i\leq n-1 \Rightarrow y_n'=-a_ny_1-a_{n-1}y_2-..-a_1y_n$
\end{document}