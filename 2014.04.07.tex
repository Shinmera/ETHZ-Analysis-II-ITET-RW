\documentclass[12pt,a4paper,titlepage]{article}
\usepackage[utf8]{inputenc}
\usepackage[T1]{fontenc}
\usepackage[top=3cm, bottom=3cm, left=2cm, right=2cm]{geometry}
\usepackage{textcomp}
\usepackage{amsmath}
\usepackage{mathtools}
\usepackage{amsfonts}
\usepackage{amssymb}
\usepackage{amsthm}
\usepackage{titlesec}
\usepackage{fancyhdr}
\usepackage{lastpage}
\usepackage{fix-cm}
\usepackage{graphicx}
\usepackage{hyperref}
\usepackage{xcolor}
\usepackage{mdwlist}
\usepackage{listings}
\usepackage{float}
\usepackage{wrapfig}
\usepackage{datetime}
\usepackage[perpage,para,bottom,marginal]{footmisc}
\usepackage{listings}
\usepackage{caption}
\usepackage{enumitem}
\usepackage{multicol}
\usepackage[cmtip,all]{xy}
\newdateformat{dmny}{\monthname[\THEMONTH] \THEYEAR}
\newdateformat{dyo}{\THEYEAR}
\setlength{\headheight}{30pt}
\pagestyle{fancy}

\author{Nicolas Hafner}
\lhead{Nicolas Hafner}
\title{Analysis II}
\chead{Analysis II}
\rhead{Zürich, \dmny\today}
\cfoot{\thepage\ / \pageref{LastPage}}
\lfoot{\copyright \dyo\today TymoonNET/NexT}
\date{\d_mny\today}

\newcommand{\longsquiggly}{\xymatrix{{}\ar@{~>}[r]&{}}}
\renewcommand{\Re}{\operatorname{Re}}
\renewcommand{\Im}{\operatorname{Im}}
\renewcommand{\arg}{\operatorname{arg}}
\renewcommand{\d}{\partial}
\newcommand{\arsinh}{\operatorname{arsinh}}
\newcommand{\arcosh}{\operatorname{arcosh}}
\newcommand{\artanh}{\operatorname{artanh}}
\newcommand{\setC}{\mathbb{C}}
\newcommand{\setR}{\mathbb{R}}
\newcommand{\setZ}{\mathbb{Z}}
\newcommand{\setN}{\mathbb{Z}^{\geq0}}
\newcommand{\Graph}{\operatorname{Graph}}
\newcommand{\vol}{\operatorname{vol}}
\newcommand{\diam}{\operatorname{diam}}

\newcount\colveccount
\newcommand*\colvec[1]{
  \global\colveccount#1
  \begin{pmatrix}
    \colvecnext
  }
  \def\colvecnext#1{
    #1
    \global\advance\colveccount-1
    \ifnum\colveccount>0
    \\
    \expandafter\colvecnext
    \else
  \end{pmatrix}
  \fi
}

\begin{document}	
\begin{center}{\bfseries\Huge Analysis II - 2014.04.07}\end{center}
\section*{Integral mehrerer Variablen}
\textit{Fakt}: Jede $\overbrace{\text{hinreichend gutartige}}^{\mathclap{\text{z.B. jede beschränkte offene oder abgeschlossene Teilmenge.}}}$ Teilmenge $X\subset\setR^n$ hat ein Volumen $\vol_n(X)\in\setN\cup\{\infty\}$. $n=1$: Länge $n=2$: Flächeninhalt. \\
\\
\textit{Eigenschaften}\begin{itemize}
\item invariant unter Translationen, Drehungen, Spiegelungen.
\item $\vol_{m+n}(X\times Y)=\vol_m(X)\cdot\vol_n(Y)$ für $X\subset\setR^m,\; Y\subset\setR^n$
\item $\vol_n(X\cup Y)=\vol_n(X)+\vol_n(Y)-\vol_n(X\cap Y)$
\item $\vol_n(X)=0$ für jedes $X$ der Dimension $< n$.
\item $\vol_n(X)\leq\vol_n(Y)$ für alle $X\subset Y$.
\item Zu pt. 3\&4: Insbesondere gilt $\vol_n(X\cup Y)=\vol_n(X)+\vol_n(Y)$ wenn $X\cap Y$ Dimension $<n$ hat.
\end{itemize}
vorläufig: $\vol:=\vol_n$
\section*{Riemann Integral}
\textit{Zerlegung}: $Z$ von $X\subset\setR^n$ in endlich viele: $X=\bigcup\limits_{i=1}^r X_i$ so dass $\vol(X_i\cap X_j)=0$ für alle $i\neq j$ mit $x_i\in X_i$ Basispunkt. \\
\\
Durchmesser von $X_i$: $\diam(X_i):=\sup(|x-y|:x,y\in X_i)$\\
\\
Feinheit von $Z$ ist $\delta(Z):=\max\limits_{i=1..r}\{\diam(X_i)\}$ \\
\\
$f$ Funktion auf $X: S_f(Z):=\sum\limits_{i=1}^rf(x_i)\vol_n(X_i)$\\
\\
Riemannsumme: $\int_X f(x)\; d\vol_n(x):=\lim\limits_{\mathclap{\substack{\text{Z Zerlegung von X}\\\delta(Z)\to 0}}}S_f(Z)$ \quad falls der Grenzwert existiert. \\
\\
\textit{Satz}: Jede stetige Funktion auf einer kompakten Teilmenge ist Riemann-integrierbar. \\
\\
\textit{Spezialfall}: $X=[a,b]\subset\setR \longsquiggly \int\limits_Xf(x) \;d\vol_1(x)=\int\limits_a^bf(x) \;dx$ \\
\\
\textit{Alternative Bezeichnungen}: $d\mu(x),\;\mu(x),\;\overbrace{dF}^{\mathclap{\dim 2}},\;\overbrace{dV}^{\mathclap{\dim 3}}$ 
\section*{Eigenschaften}
$\int_X1 \;d\vol_n=\vol_n(X)$ \\
$\int_{X\cup Y}f \;d\vol_n=\int_xf \;d\vol_n+\int_Yf \;d\vol_n$ falls $X\cap Y$ Dimension $<n$ hat. \\
$\int_X(\lambda_1f_1+\lambda_2f_2) \;d\vol_n=\lambda_i\int_Xf_1 \;d\vol_n+\lambda\int_Xf_2 \;d\vol_n$ \\
$\int_Xf \;d\vol_n\leq \int_Xg \;d\vol_n$ falls überall $f\leq g$. \\
$\left|\int_Xf \;d\vol_n\right|\leq \int_X\left|f\right| \;d\vol_n$ \\
\\
\textit{Satz (Fubini)}: Sei $X\subset\setR^n$ kompakt, seine $\varphi,\psi:X\to\setR$ stückweise stetig mit $\forall x:\varphi(x)\leq\psi(x)$ \\
$Z:=\left\{\colvec{2}{x}{y}\in\setR^{n+1}\mid\substack{x\in X,\; y\in\setR \\ \varphi(x)\leq y\leq \psi(x)}\right\}$ Sei $f$ stetig auf $Z$, dann
$\int\limits_Zf \;d\vol_{n+1}=\int\limits_X\left(\int\limits_{\varphi(x)}^{\psi(x)}f\colvec{2}{x}{y} \;dy\right) \;d\vol_n(x)$ \\
\\
\textit{Speziell}: $\vol(Z)=\int_X(\psi(x)-\varphi(x)) \;d\vol_n(x)$ \\
\\
\textit{Beispiel}: $B:=\left\{\colvec{2}{x}{y}\in\setR^2\mid\substack{0\leq x\leq 4 \\ -\sqrt{x}\leq y\leq\sqrt{x}}\right\}=\left\{\colvec{2}{x}{y}\in\setR^2\mid\substack{-2\leq y\leq 2 \\ y^2\leq x\leq 4}\right\}$ \\
$\int_Bf\colvec{2}{x}{y} \;d\vol\colvec{2}{x}{y} = \int_0^4(\int_{-\sqrt{x}}^{\sqrt{x}}f\colvec{2}{x}{y} \;dy) \;dx = \int_{-2}^2(\int_{y^2}^4f\colvec{2}{x}{y} \;dx) \;dy$

\end{document}